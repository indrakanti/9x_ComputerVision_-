\chapter*{Index of Chapters}
\addcontentsline{toc}{chapter}{Index of Chapters}

\noindent
This book unfolds as a continuous journey --- from the physics of light to the intelligence of machines. 
Each chapter is crafted to build upon the previous, uniting mathematics, perception, and engineering into one framework for vision.

\vspace{1em}

\begin{description}[leftmargin=3em,style=nextline]

%---------------------------------------------------------------
\item[\textbf{Part I – Foundations of Intelligent Vision}]
\textit{The physical, mathematical, and computational roots of how machines see.}

\begin{itemize}[leftmargin=2em]
  \item \textbf{Chapter 0 – Foundations Overview:} Why vision matters, and the grand map --- light $\rightarrow$ signal $\rightarrow$ math $\rightarrow$ vision $\rightarrow$ AI.
  \item \textbf{Chapter 1 – The Mathematics of Seeing:} Light as measurable energy; vectors, geometry, and optimization as the language of perception.
  \item \textbf{Chapter 2 – The Physics and Perception of Vision:} Optics, sensors, and the human analogy — connecting mathematical models with physical reality.
  \item \textbf{Chapter 3 – From Equations to Algorithms:} Turning formulas into filters, convolutions, and frequency-domain reasoning.
  \item \textbf{Chapter 4 – Building the Computational Pipeline:} Engineering the end-to-end system — from capture to display, ready for learning.
\end{itemize}

\vspace{1em}

%---------------------------------------------------------------
\item[\textbf{Part II – Learning from Data}]
\textit{Where perception meets data-driven inference.}

\begin{itemize}[leftmargin=2em]
  \item \textbf{Chapter 5 – Learning in the Feature Space:} From handcrafted features to clustering and classification using PCA, LDA, and SVMs.
  \item \textbf{Chapter 6 – Vision as Inference:} Probabilistic models, Kalman and particle filters, and reasoning under uncertainty.
\end{itemize}

\vspace{1em}

%---------------------------------------------------------------
\item[\textbf{Part III – Deep Learning for Vision}]
\textit{From features to learned representations and semantic understanding.}

\begin{itemize}[leftmargin=2em]
  \item \textbf{Chapter 7 – Convolutional Intelligence:} CNNs, receptive fields, and transfer learning.
  \item \textbf{Chapter 8 – Detect, Segment, and Understand:} Object detection, segmentation, and explainability through attention.
\end{itemize}

\vspace{1em}

%---------------------------------------------------------------
\item[\textbf{Part IV – Advanced and Frontier Topics}]
\textit{Efficiency, fusion, and the newest paradigms in visual AI.}

\begin{itemize}[leftmargin=2em]
  \item \textbf{Chapter 9 – Fusion, Efficiency, and Edge AI:} Multimodal sensing, lightweight models, quantization, and deployment.
  \item \textbf{Chapter 10 – The Frontier of Vision: Transformers and Beyond:} Vision Transformers, diffusion models, foundation models, and ethics.
\end{itemize}

\vspace{1em}

%---------------------------------------------------------------
\item[\textbf{Part V – Engineering Labs and Real-World Systems}]
\textit{Applying it all — full-stack perception projects and embedded implementations.}

\begin{itemize}[leftmargin=2em]
  \item \textbf{Chapter 11 – Applied Projects in Modern Vision Systems:} End-to-end projects — ADAS, thermal fusion, industrial inspection, ophthalmic AI, robotics, and generative vision.
\end{itemize}

\end{description}

\vspace{2em}
\noindent
Each part concludes with a practical \textit{Code Lab} and references to reproducible Git repositories (see Appendix D).
Together, these chapters form a complete pathway from physics and mathematics to modern AI-powered perception.
